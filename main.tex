\documentclass[a4paper, 11pt, twocolumn, final]{article}

\usepackage[IL2]{fontenc}
\usepackage{times}
\usepackage[utf8]{inputenc}
\usepackage[czech]{babel}
\usepackage[left=1.5cm, text={18cm, 25cm}, top=2.5cm]{geometry}

\usepackage[unicode, hidelinks]{hyperref}
\usepackage{amsmath}
\usepackage{amsthm}
\usepackage{amssymb}
\usepackage{stackrel}
\usepackage{enumitem}

\parindent=1em

\theoremstyle{definition}
\newtheorem{definition}{Definice}

\theoremstyle{plain}
\newtheorem{theorem}{Věta}

\begin{document}

\begin{titlepage}
\begin{center}
    {\Huge \textsc{Fakulta informačních technologií}\\[0.4em]}
    {\Huge \textsc{Vysoké učení technické v~Brně}}\\
    
    \vspace{\stretch{0.382}}
    
    {\LARGE Typografie a publikování\,--\,2. projekt\\[0.3em]}
    {\LARGE Sazba dokumentů a matematických výrazů}\\
    
    \vspace{\stretch{0.618}}
\end{center}

{\Large \noindent 2021 \hfill Michal Šmahel (xsmahe01)}
\end{titlepage}

\section*{Úvod}

V~této úloze si vyzkoušíme sazbu titulní strany, matematic\-kých vzorců, prostředí
a dalších textových struktur obvyklých pro technicky zaměřené texty (například
rovnice (\ref{eq:bracket-types}) nebo Definice \ref{def:expanded-stack-automat}
na straně \pageref{def:expanded-stack-automat}). Rovněž si vyzkoušíme používání
odkazů \verb|\ref| a \verb|\pageref|.

Na titulní straně je využito sázení nadpisu podle optického středu s~využitím
zlatého řezu. Tento postup byl probírán na přednášce. Dále je použito odřádkování
se zadanou relativní velikostí 0.4\,em a 0.3\,em.

V~případě, že budete potřebovat vyjádřit matematickou
konstrukci nebo symbol a nebude se Vám dařit jej nalézt
v~samotném \LaTeX{}u, doporučuji prostudovat možnosti balíku maker \AmS-\LaTeX.

\section{Matematický text}

Nejprve se podíváme na sázení matematických symbolů a~výrazů v~plynulém textu
včetně sazby definic a vět s~využitím balíku \texttt{amsthm}. Rovněž použijeme
poznámku pod čarou s~použitím příkazu \verb|\footnote|. Někdy je vhodné použít
konstrukci \verb|\mbox{}|, která říká, že text nemá být zalomen.

\begin{definition}
    \label{def:expanded-stack-automat}
    Rozšířený zásobníkový automat \emph{(RZA) je definován jako sedmice tvaru
    $A = (Q, \Sigma, \Gamma, \delta, q_0, Z_0, F)$, kde:}

    \begin{itemize}[label=$\bullet$]
        \item $Q$ \emph{je konečná množina} vnitřních (řídicích) stavů,
        \item $\Sigma$ \emph{je konečná} vstupní abeceda,
        \item $\Gamma$ \emph{je konečná} zásobníková abeceda,
        \item $\delta$ \emph{je} přechodová funkce $Q \times (\Sigma \cup
        \{\epsilon\}) \times \Gamma^\ast\ {\rightarrow}\ 2^{Q \times
        \Gamma^\ast}$,
        \item $q_0 \in Q$ \emph{je} počáteční stav, $Z_0 \in \Gamma$ \emph{je}
        startovací symbol zásobníku a $F \subseteq Q$ \emph{je množina}
        koncových stavů.
    \end{itemize}
    
    Nechť $P = (Q, \Sigma, \Gamma, \delta, q_0, Z_0, F)$ je rozšířený
    zásobníkový automat. \emph{Konfigurací} nazveme trojici $(q, w, \alpha)
    \in Q \times \Sigma^\ast \times \Gamma^\ast$, kde $q$ je aktuální stav
    vnitřního řízení, $w$ je dosud nezpracovaná část vstupního řetězce
    a $\alpha = Z_{i_1} Z_{i_2} \ldots Z_{i_k}$ je obsah zásobníku\footnote{
    $Z_{i_1}$ je vrchol zásobníku}.
\end{definition}

\subsection{Podsekce obsahující větu a odkaz}

\begin{definition}
    \label{def:string-w}
    Řetězec $w$ nad abecedou $\Sigma$ je přijat RZA $A$ jestliže $(q_0, w, Z_0)
    \stackrel[A]{\ast}{\vdash} (q_F, \epsilon, \gamma)$ \emph{pro nějaké}
    $\gamma \in \Gamma^\ast$ a $q_F \in F$. \emph{Množinu} $L(A) = \{w\ |\ w$
    \emph{je přijat RZA} $A\} \subseteq \mbox{$\Sigma^\ast$
    \emph{nazýváme} jazyk přijímaný RZA\ $A$}$.
\end{definition}

Nyní si vyzkoušíme sazbu vět a důkazů opět s~použitím balíku \texttt{amsthm}.

\begin{theorem}
    \label{th:lang-class}
    Třída jazyků, které jsou přijímány ZA, odpovídá {\normalfont bezkontextovým
    jazykům}.
\end{theorem}

\begin{proof}
    V~důkaze vyjdeme z~Definice \ref{def:expanded-stack-automat}
    a \ref{def:string-w}.
\end{proof}

\section{Rovnice a odkazy}

Složitější matematické formulace sázíme mimo plynulý text. Lze umístit několik
výrazů na jeden řádek, ale pak je třeba tyto vhodně oddělit, například
příkazem \verb|\quad|.

$$\sqrt[i]{x_{i}^3} \quad \text{kde}\ x_i\ \text{je}\ i\text{-té sudé číslo splňující}
\quad x_{i}^{x_{i}^{i^2} + 2} \leq y_{i}^{x_{i}^4}$$

V~rovnici (\ref{eq:bracket-types}) jsou využity tři typy závorek s~různou
explicitně definovanou velikostí.

\begin{eqnarray}
    \label{eq:bracket-types}
    x &= & \left[\Big\{\big[a + b\big] \ast c\Big\}^d \oplus 2\right]^{3/2}\\
    \nonumber y &= & \lim_{x \rightarrow \infty} \frac{\frac{1}{\log_{10} x}}
    {\sin^2 x + \cos^2 x}
\end{eqnarray}

V~této větě vidíme, jak vypadá implicitní vysázení limity $lim_{n \rightarrow
\infty} f(n)$ v~normálním odstavci textu. Podobně je to i s~dalšími symboly
jako $\prod_{i = 1}^n 2^i$ či $\bigcap_{A \in \mathcal{B}} A$. V~případě
vzorců $\lim\limits_{n \rightarrow \infty} f(n)$ a $\prod\limits_{i = 1}^n 2^i$
jsme si vynutili méně úspornou sazbu příkazem \verb|\limits|.

\begin{equation}
    \label{eq:integrals}
    \int_b^a g(x)\mathrm{d}x = -\int\limits_a^b f(x)\mathrm{d}x
\end{equation}

\section{Matice}

Pro sázení matic se velmi často používá prostředí \texttt{array} a závorky
(\verb|\left|, \verb|\right|).

$$\left( \begin{array}{ccc}
         a - b   & \widehat{\xi + \omega}  & \pi \\
         \vec{\mathbf{a}} & \overleftrightarrow{AC} & \hat{\beta}
    \end{array} \right) = 1 \Longleftrightarrow \mathcal{Q} = \mathbb{R}$$

$$A = \left\| \begin{array}{cccc}
        a_{11} & a_{12} & \ldots & a_{1n} \\
        a_{21} & a_{22} & \ldots & a_{2n} \\
        \vdots & \vdots & \ddots & \vdots \\
        a_{m1} & a_{m2} & \ldots & a_{mn}
    \end{array} \right\| = \left| \begin{array}{rl}
        t & u \\
        v & w
    \end{array} \right| = tw {-} uv$$

Prostředí \texttt{array} lze úspěšně využít i jinde.

$$\binom{n}{k} = \left\{ \begin{array}{cl}
        0                   & \ \text{pro}\ k < 0\ \text{nebo}\ k > n \\
        \frac{n!}{k!(n-k)!} & \ \text{pro}\ 0 \leq k \leq n.
    \end{array} \right.$$
\end{document}
